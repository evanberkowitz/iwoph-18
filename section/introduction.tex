\section{Introduction}

The computational resources provided by the CORAL supercomputers provides challenges for efficient use.
Some applications may need to scale to an \order{1} fraction of the whole machine for a single run of an executable, while others may need a variety of executables excuting heterogenous tasks that require different computational resources.  One example is lattice field theory, the numerical prescription for solving quantum field theories.

We typically wish to study quantum field theories in continuum spacetime; with lattice methods we discretize spacetime and extrapolate to take the continuum limit.
Computations are typically done in spacetime boxes with periodic boundary conditions, while we are interested in extrapolating to the infinite-volume result.
Finally, the computational physicist may adjust the input parameters to the theory, if the physical parameters are too computationally costly, and attempt to extrapolate or interpolate to the true parameters.

Each choice of discretization scale, volume size, and input parameters requires the generation of a stochastic ensemble of the spacetime configuration of the quantum field via importance sampling and `measurements' on each configuration in the ensemble.
The configurations are stored and can be reused when computing other observables.

The measurement step frequently requires linear solves of the \emph{Dirac matrix}.
In the case of QCD, the theory of quarks and gluons, the Dirac matrix is a discretization-dependent (typically nearest-neighbor) stencil operator whose linear size is the number of spacetime points in the configuraiton.
Each entry in that matrix is a $12\times12$ matrix which describes how quantities should be parallel transported around the lattice\footnote{12=(4 fermionic spin degrees of freedom)$\times$(3 fermionic \emph{color} degrees of freedom)}.
Solving this linear system is often accomplished with numerical acceleration; the community library for nVidia GPUs is \quda\cite{Clark:2009wm,Babich:2011np}.

We call the solutions \emph{propagators}, as they describe how quarks, described initially by the RHS of the solve, evolve.
These propagators can be reused to compute many observables, and too are often stored.
Different observables require different contractions.


Our previous publications are \cite{Berkowitz:2017vcp,Berkowitz:2017xna}.
